\documentclass[12pt]{article}

\usepackage[
   paperheight = 11in,
   paperwidth = 8.5in,
   margin = 1in,
   footskip = 0.45in] {geometry} %  custom paper size and margins
\usepackage{graphicx}
\usepackage{float}
\usepackage{adjustbox}
\usepackage{color} %  custom colors
\usepackage{multicol}
\usepackage{enumitem}
\usepackage{booktabs}
\usepackage{hhline}
\usepackage{titlesec}
\usepackage{hyphenat}
\usepackage{array}
\usepackage{graphicx}
\usepackage{listings, upquote, textcomp} %  "\lstinline{ }" for inline code
\usepackage[
   colorlinks = true,
   linkcolor = darkblue,
   urlcolor = darkblue,
   hypertexnames = false]{hyperref}

\setlength{\tabcolsep}{10pt}
\def\arraystretch{1.2}
\definecolor{gray}         {rgb}{0.50,0.50,0.50}
\definecolor{darkblue}     {rgb}{0.00,0.00,0.50}
\renewcommand{\labelenumi}{\arabic{enumi}.}
\renewcommand{\labelenumii}{\alph{enumii}.}
\setlength{\parindent}{0em}
\setlength{\parskip}{10pt plus 2pt minus 2pt}
\titlespacing\section{0pt}{10pt plus 2pt minus 2pt}{0pt plus 2pt minus 2pt}
\raggedbottom
\newcolumntype{L}[1]{>{
   \raggedright\let\newline\\\arraybackslash\hspace{0pt}}p{#1}}
\newcolumntype{C}[1]{>{
   \centering\let\newline\\\arraybackslash\hspace{0pt}}p{#1}}
\newcolumntype{R}[1]{>{
   \raggedleft\let\newline\\\arraybackslash\hspace{0pt}}p{#1}}
\lstdefinestyle{lstMat} {
   language = matlab,
   backgroundcolor = {},
   breakatwhitespace = true,
   breaklines = true,
   postbreak = \space,
   breakindent = 4ex,
   showstringspaces = false,
   basicstyle = \footnotesize\ttfamily,
   commentstyle = \color{gray},
   stringstyle = \color{red},
   keywordstyle = \color{blue},
   tabsize = 3,
   upquote = true}
\lstset{style = lstMat} %  Default syntax.

\begin{document}
%  Header
\begin{minipage}{0.9\textwidth}
   \raggedright
   \large \textbf{\textsf{{\color{gray}EENG 533: Navigation Using GPS} \\
      Project 2: Calculating GPS Satellite Position}}
\end{minipage}
\vspace{1cm}

\section*{\textsf{Objectives}}

\begin{enumerate}
   \item Become more familiar with Python programming and explore the 
            \href{https://pypi.org/project/georinex/}{georinex} RINEX file processing library.
   \item Develop a routine to determine satellite position, which will be used
      throughout the rest of the course.
   \item Develop a better understanding of coordinate frames and timing
      issues.
\end{enumerate}

\section*{\textsf{Overview}}

You will be calculating satellite positions using broadcast ephemeris data
from \href{https://en.wikipedia.org/wiki/RINEX}{RINEX} (Receiver Independent Exchange Format) 
file throughout this course, and the purpose of this lab is to develop a tool, a Python function, 
for doing this.  You will write a Python function which can be
called to calculate the coordinates of the satellite for use in positioning
applications.  It will also calculate the satellite clock error, which will be
needed in future labs.  The equations needed are found on pp. 95-96, and 102-103 of
document \href{https://www.gps.gov/technical/icwg/IS-GPS-200L.pdf}{IS-GPS-200L}.

\section*{\textsf{Collaboration}}

This is an individual lab.   You are allowed to discuss any aspect of the lab
with other students, and you may look at each other's source code for debugging
purposes.  However, your programming must be your own; you may not copy or
transcribe someone else's program, in part or in whole.

\section*{\textsf{Satellite Position Function Description}}

Your assignment is to write the following MATLAB function

% \begin{lstlisting}[style = lstMat]
\begin{lstlisting}[language = Python]
import math
import numpy as np


def calc_sv_pos(eph, tow, rcvr_ecef):
    """
    Satellite position and satellite clock correction calculation based on IS-GPS-200L (Table 20-IV and pp. 95-96).
    The satellite position is for a particular GPS time, expressed in the ECEF coordinate frame valid at the time the signal was received by a GPS receiver.

    :param eph:              ephemeris object (from georinex) for a single satellite and time epoch
    :param tow:              time of transmission [GPS week sec]
    :param rcvr_ecef:        1x3 numpy array of ECEF receiver position [x y z in meters]

    :return sv_ecef:         1x3 numpy array of ECEF satellite position [x y z in meters] at time of transmission, expressed in ECEF frame at the time of reception
    :return sv_clock_error:  satellite clock error [sec] at time of transmission
    """
    # <<<your code here>>>

     return sv_ecef, sv_clock_error
\end{lstlisting}

%\noindent If you copy the source as given directly above, then you will have a
%useful output when you type \lstinline{help calc_sv_pos} in MATLAB.

The algorithms for calculating the ECEF satellite position are provided in
IS-GPS-200L (Table 20-IV), which can be obtained from \href{https://www.gps.gov/technical/icwg/IS-GPS-200L.pdf}{this link}.
In addition, you should reference the class slide handouts
to get additional information that you will need to complete this lab.  Note
that the coordinates calculated by the equations in IS-GPS-200L ($x_k$, $y_k$,
and $z_k$) are the ECEF coordinates of the satellite at the time of transmission
(shown as $x_t$, $y_t$, and $z_t$ in the slide titled ``Coordinate Frame
Rotation'' in the fourth video).  You will need to perform the rotation to
express them as the ECEF coordinates at the time of reception ($x_r$, $y_r$, and
$z_r$).

The algorithms for calculating the SV clock correction ($\Delta t_{sv}$) are
found in the IS-GPS-200L section entitled ``User Algorithm for SV Clock
Correction'' (pp. 95-96).  Note that you need the eccentric anomaly (an intermediate
calculation when calculating the position) in order to do this.  This SV clock
correction represents a significant error that needs to be removed if anything
useful is done with GPS measurements.

\section*{\textsf{Obtaining the Ephemeris Data}}

The raw ephemeris data is in a provided text file called \texttt{ohdt3490.20n}.  
%The MATLAB function \texttt{load\_eph\_file} will read all the data from the
%\texttt{.eph} file into a global variable.  The MATLAB function
%\texttt{current\_ephemeris} will select the data of a particular satellite,
%identified by its PRN number.

%The idea is that before you run your MATLAB program, you call
%\texttt{load\_eph\_file} once from the command window.  Then, each time you need
%an ephemeris, you retrieve the appropriate ephemeris record using the
%\texttt{current\_ephemeris} function from within your \texttt{calc\_sv\_pos}.
%Use the help function for more information about these functions.

The georinex library and functions in the provided \texttt{helper.py} file will perform the work needed
to isolate the correct ephemeris record from the RINEX nav file.  The provided
\texttt{project2\_template.py} file contains an example for a specific PRN and transmission time.

The georinex library needs to be added to your Anaconda environment.  Unfortunately, it is not
contained in the Anaconda repository so you will need to choose the ``CMD.exe Prompt" (or similar
depending on your operating system) application and type \texttt{pip install georinex} at the 
prompt.  It may take a few minutes to install.  Typing \texttt{pip list} will show all the installed libraries in your environment 
and georinex should now be listed.

The \texttt{ephem} object created near the end of \texttt{project2\_template.py} from the class definition 
in \texttt{ephemeris.py} will contain the ephemeris values for the chosen satellite 
and for the time closest to the \texttt{tow} specified.  When implementing the equations in your
function to calculate satellite position and clock error, the ephemeris values can be accessed as
\texttt{eph.M\_0}, \texttt{eph.e}, etc.  The full listing of parameters is shown in \texttt{ephemeris.py}.



\section*{\textsf{Checking Your Work}}

Below is a sample run from a satellite positioning algorithm that you can use to
check your work.

\begin{verbatim}
   Filename       : ohdt3490.20n
   PRN            : G03
   Transmit Time  : 80000.000
   Rcvr Position  : -1485881.487 -5152018.353 3444641.847 (meters)
   SV Position    : -12334966.483 -22955725.027 -5479289.332 (meters)
   SV Clock Error : -0.0000290720 (seconds)
\end{verbatim}

Hint: Here is a sample command that would generate some of the above printout.
You should not have this in your function, but you could use something like it
in a test program which would call your function.
\begin{lstlisting}[language = Python]
    print("SV Position    : %.3f %.3f %.3f (meters)" % (sv_ecef[0], sv_ecef[1], sv_ecef[2]))
\end{lstlisting}

\section*{\textsf{Deliverables}}

For this project, please submit your function as a \texttt{.py} file in Canvas.
Please name it
\begin{verbatim}
   calc_sv_pos_<lastname>.py
\end{verbatim}
where \texttt{<lastname>} is your last name in lower case.  If John Smith were
turning in the lab, the filename would be \texttt{calc\_sv\_pos\_smith.py}.

Note that I only want your function so that I can call it myself, not a program
that will generate an output similar to that shown in the ``Checking Your Work''
section above.  In fact, your function should not print out anything at all: it
should just return the output vectors.

\section*{\textsf{Grading}}

You will be graded primarily upon accuracy.  If you get the correct positions
for satellites and times that I pick when I run your function, then you will get
full credit for the lab.  I will try several different PRNs at different times.
If you do not get the correct answers, then I will look at your source code and
make a judgment on your grade based on what I see.  Feel free to include any
comments that you would like in Canvas.  Also, make sure that your function
works with a 1-by-3 \texttt{rcvr\_ecef} vector and returns a 1-by-3
\texttt{sv\_ecef} vector.  Points will be deducted if either of these is wrong.
Also, make sure that your function can handle end of week rollover (as described
in the IS-GPS-200L) for both the satellite position and clock error
calculations.  Failure to do so will result in points being deducted for each
case in which the error is made (position and clock error).

\end{document}
